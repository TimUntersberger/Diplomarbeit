\chapter{Resümee}

\section{Tim Untersberger}

Im Laufe dieser Arbeit habe ich das Arbeiten mit Mikrocontrollern näher kennengelernt und meine C/C++ verbessern dürfen.

Da die Entwicklung mit Mikrocontrollern neu für mich war, erwies sich der Beginn der Arbeit als holprig. Durch mein kontinuierlich wachsendes Verständnis über das IOT-Ökosystem machte mir das Programmieren mehr und mehr Spaß.

Da ich bis jetzt hauptsächlich Projekte alleine entwickelt habe, fiel mir die Zusammenarbeit mit einem Partner anfangs eher schwer. Dank fortwährender Kommunikation konnten wir diese Hürde als Team meistern.

Durch diese Arbeit habe ich wichtige Grundsätze für die Zukunft meiner Softwareenwickler Karriere gelernt.

\section{Stefan Waldl}
Beim Arbeiten an dieser Diplomarbeit konnte ich viel für mein Leben lernen. Der größte Punkt, was das betrifft, war das Arbeiten mit einem Partner. Man lernt, wie wichtig es ist einfach aber trotzdem prägnant zu kommunizieren, da sonst bei einem größeren Projekt wie diesem leicht Missverständnise auftreten können.

Anfangs war es sehr mühsam, auf Mikrocontrollern zu programmieren, da sich diese ganz anders verhalten als herkömmliche Computer. Doch als ich mich an die Eigenheiten der Mikrocontrollern gewöhnt hatte, war es mir möglich, in einen produktiven Workflow zu gelangen.

Im Großen und Ganzen bin ich froh, mein Know-How um die Welt der Mikrocontroller erweitern haben zu können.