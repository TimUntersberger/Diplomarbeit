\chapter{Resümee}
%TODO technisch ziele, was wurde erreicht
%todo hardware anders zu arbeiten

\section{Tim Untersberger}
%todo ändern

Mein Partner und ich wussten am Anfang nicht welches Thema wir für unsere Diplomarbeit wählen. Zum Glück haben uns Herr Prof. Stütz und Herr Prof. Köck dieses Thema vorgeschlagen. Im Laufe dieser Arbeit habe ich das Arbeiten mit Mikrocontrollern näher kennengelernt und meine C/C++ verfeinern dürfen. 

Das Zusammenarbeiten mit einem Partner war am Anfang eine kleine Herausforderung, welche aber zum Glück schnell Überwunden wurde.

Diese Diplomarbeit hat mir einige Dinge im Thema Zeitmanagement beigebracht und ich werde diese Erfahrung mein Leben lang schätzen.

Das Schreiben der Arbeit selber war für mich die größte Herausforderung, da ich davor Doku relevanten Dinge meistens vernachlässigt bzw. ausgelassen habe.

\section{Stefan Waldl}
Beim Arbeiten an dieser Diplomarbeit konnte ich viel für mein Leben lernen. Der größte Punkt, was das betrifft, war das Arbeiten mit einem Partner. Man lernt, wie wichtig es ist einfach aber trotzdem prägnant zu kommunizieren, da sonst bei einem größeren Projekt wie diesem leicht Missverständnise auftreten können.

Anfangs war es sehr mühsam, auf Mikrocontrollern zu programmieren, da sich diese ganz anders verhalten als herkömmliche Computer. Doch als ich mich an die Eigenheiten der Mikrocontrollern gewöhnt hatte, war es auch mir möglich, in einen produktiven Workflow zu gelangen.

Im Großen und Ganzen bin ich froh, mein Know-How um die Welt der Mikrocontroller erweitern haben zu können.