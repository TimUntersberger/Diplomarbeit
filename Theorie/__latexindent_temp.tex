\chapter{Arbeitsaufteilung}

\section*{Tim Untersberger}

\begin{itemize}
	\item Zusammenfassung
	\item Autoren der Diplomarbeit
	\begin{itemize}
		\item Tim Untersberger
	\end{itemize}
	\item Danksagung
	\item Verwendete Technologien
	\begin{itemize}
		\item Message Queuing Telemetry Transport (MQTT)
		\begin{itemize}
			\item QoS
		\end{itemize}
		\item Docker
		\begin{itemize}
			\item Docker vs. Virtuelle Maschinen
			\item Begriffe
		\end{itemize}
		\item Docker Compose
		\begin{itemize}
			\item Docker Compose Beispiel
			\begin{itemize}
				\item Postgres
				\item PgAdmin4
			\end{itemize}
		\end{itemize}
	\end{itemize}
	\item Ausgewählte Aspekte
	\begin{itemize}
		\item Mesh Visualizer
		\begin{itemize}
			\item Cytoscape
			\begin{itemize}
				\item Dagre
				\item Euler
				\item Spread
			\end{itemize}
		\end{itemize} 
		\item Libraries
		\begin{itemize}
			\item Eigene Libraries
			\begin{itemize}
				\item MQTT
				\item Hypertext Transfer Protocol (HTTP)
				\item Mesh
			\end{itemize}
		\end{itemize}
		\item Erläuterung der Verwendung des Mesh-Netzwerkes
		\begin{itemize}
			\item Übersicht
			\item Nodejs Setup
			\item ESP-IDF Setup
			\begin{itemize}
				\item Dependencies
				\item Libraries
				\item Umgebungsvariablen
				\item WSL
			\end{itemize}
			\item Source Code
			\item Mosquitto
			\item Mesh Visualizer
			\item Partition Table
			\item Config
			\begin{itemize}
				\item Example Configuration
				\item Serial flasher config
				\item Partition Table
			\end{itemize}
			\item Flashen
			\item Ergebnis
			\begin{itemize}
				\item Mesh Visualizer
				\item Mqtt Fx
			\end{itemize}
			\item Code
			\begin{itemize}
				\item Struktur
				\item Setup
				\item Wifi Setup
				\item IP Event Handler
				\item Command Callback
				\item DHT22 Task
			\end{itemize} 
		\end{itemize} 
	\end{itemize}
	\item Resümee
	\begin{itemize}
		\item Tim Untersberger
	\end{itemize}
\end{itemize}

\section*{Stefan Waldl}

\begin{itemize}
	\item Zusammenfassung
	\item Autoren der Diplomarbeit
	\begin{itemize}
		\item Stefan Waldl
	\end{itemize}
	\item Danksagung
	\item Ausgangssituation und Zielsetzung
	\begin{itemize}
		\item Ausgangssituation
		\item Beschreibung des Problembereichs
		\item Aufgabenstellung
		\item Zielbestimmung
	\end{itemize}
	\item Verwendete Technologien
	\item ESP-IDF Toolchain
	\begin{itemize}
		\item KConfig.projbuild
		\item Kompilieren
		\item Flashen
		\begin{itemize}
			\item IDF Monitor
		\end{itemize}
		\item Platform IO
		\item Utility lib Köck
	\end{itemize}
	\item Ausgewählte Aspekte
	\begin{itemize}
		\item Verwendete Hardware
		\begin{itemize}
			\item DHT22
			\begin{itemize}
				\item OTP Memory
			\end{itemize}
			\item NodeMCU ESP32
			\begin{itemize}
				\item ESP-WROOM-32
			\end{itemize}
		\end{itemize}
		\item Over The Air Update (OTA)
		\begin{itemize}
			\item Problemstellung
			\item Wie OTA funktioniert
			\begin{itemize}
				\item Übersicht
			\end{itemize}
			\item Partition Table
			\begin{itemize}
				\item Übersicht
				\item Custom Partition Tables
				\item Name Feld
				\item Type Feld
				\item SubType
				\item Flags
			\end{itemize}
			\item OTA Partition Table
			\item ESP-MESH
			\begin{itemize}
				\item Wi-Fi vs Mesh
			\end{itemize}
		\end{itemize}
		\item Libraries
		\begin{itemize}
			\item ESP-IDF Libraries
			\begin{itemize}
				\item freertos
				\item Dht22
			\end{itemize}
		\end{itemize}
		\item Erläuterung der Verwendung des Mesh-Netzwerkes
		\begin{itemize}
			\item Hardware
			\begin{itemize}
				\item DHT22 mit ESP32 verbinden
				\item ESP32 mit Computer verbinden
			\end{itemize} 
		\end{itemize}
		\item ELF vs. Bin
		\begin{itemize}
			\item Bin
			\begin{itemize}
				\item Hintergrund
			\end{itemize}
			\item ELF
			\begin{itemize}
				\item Struktur
			\end{itemize}
			\item ELF-Header
			\begin{itemize}
				\item Class
				\item Data
				\item Version
				\item OS / ABI
				\item ABI-Version
				\item Machine
				\item Type
			\end{itemize}
			\item File Data
			\begin{itemize}
				\item Programm-Header
				\item GNUEHFRAME
				\item GNUSTACK
			\end{itemize}
			\item Static vs. Dynamic binaries
			\item Fazit
		\end{itemize}
	\end{itemize}
	\item Resümee
	\begin{itemize}
		\item Stefan Waldl
	\end{itemize}
\end{itemize}